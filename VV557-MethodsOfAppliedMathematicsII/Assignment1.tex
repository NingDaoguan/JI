\documentclass{article}
\usepackage[utf8]{inputenc}
\usepackage{geometry}
\geometry{a4paper,left=2cm,right=2cm,top=3cm,bottom=3cm}
\title{VV557-Assignment1}
\author{Group 15 \\ Yueyang Shen(517370910226), Yibo Zhao(118370910011) and Xu Zhang(118370910015)}
\date{}
\usepackage{graphicx}
\usepackage{bm}
\usepackage{amsmath,amsfonts,amsthm,amssymb}

\begin{document}
\maketitle
\section*{Exercise 1.1}
\subsection*{i)}
Solve separately in the intervals (0, $\xi-1/2n$], ($\xi-1/2n$, $\xi+1/2n$) and [$\xi+1/2n$,1):
\begin{equation}
u_n(x)=\left\{
	\begin{array}{lcl}
		Ax		&		& {0 < x \leq \xi-1/2n}\\
		-\frac{n}{2}x^2+Bx+C	&		& {\xi-1/2n < x < \xi+1/2n }\\
		D(1-x)	&		& {\xi+1/2n \leq x < 1}
	\end{array}
\right.
\end{equation}
Continuity of $u$ requires

$$
\left\{
\begin{aligned}
	& A(\xi-\frac{1}{2n})=-\frac{n}{2}(\xi-\frac{1}{2n})^2 + B(\xi-\frac{1}{2n}) + C \\
	& D(1-\xi - \frac{1}{2n})=-\frac{n}{2}(\xi+ \frac{1}{2n})^2 + B(\xi+\frac{1}{2n}) + C
\end{aligned}
\right.
$$
Continuous differentiability requires

$$
\left\{
\begin{aligned}
	& A = B - n(\xi - \frac{1}{2n}) \\
	& -D= B - n(\xi +\frac{1}{2n})
\end{aligned}
\right.
$$
Solving the above equations yields the integration constants
$$
\left\{
\begin{aligned}
	& A = 1- \xi \\
	& B = \xi (n-1) + \frac{1}{2} \\
	& C = -\frac{n}{2}(\xi-\frac{1}{2n})^2 \\
	& D = \xi
\end{aligned}
\right.
$$
Substitute the four constants into (1)

\begin{equation}
u_n(x)=\left\{
	\begin{array}{lcl}
		(1-\xi)x		&		& {0 < x \leq \xi-1/2n}\\
		(1-\xi)x -\frac{n}{2} (x-\xi + 1/2n)^2	&		& {\xi-1/2n < x < \xi+1/2n }\\
		\xi (1-x)	&		& {\xi+1/2n \leq x < 1}
	\end{array}
\right.
\end{equation}

\subsection*{ii)}
The Green function derived in the lecture is
\begin{equation}
g(x,\xi)=\left\{
	\begin{array}{lcl}
		(1-\xi)x & 0 \leq x < \xi \\
		(1-x)\xi & \xi \leq x \leq 1
	\end{array}
\right.
\end{equation}
Intervals $[0,\xi-\frac{1}{2n}]$ and $[\xi+\frac{1}{2n}]$ are easy to check. For $x\in(\xi-1/2n,\xi)$:
$$
	\lim_{n \to \infty}\left|u_n(x)-g(x,\xi)\right| = \lim_{n \to \infty}\frac{n}{2}(x-\xi+\frac{1}{2n})^2 < \lim_{n \to \infty}\left|\frac{n}{2} \cdot (\frac{1}{2n})^2\right| =0
$$
and for $x\in(\xi,\xi+\frac{1}{2n})$:
$$
	\lim_{n \to \infty}\left|u_n(x)-g(x,\xi)\right| < \lim_{n \to \infty}\frac{n}{2}(x-\xi-\frac{1}{2n})^2 <  \lim_{n \to \infty}\left|\frac{n}{2} \cdot (\frac{1}{2n})^2\right| =0
$$
As $n \to \infty$, the sequence of solutions converges point-wise on [0, 1] as $n \to \infty$ to the Green function (3).

\subsection*{iii)}
Compare the solution (2) with the Green function (3), it is obvious that only the interval $(\xi-1/2n, \xi+1/2n)$ needs to be verified
$$
\lim_{n \to \infty} \sup_{x\in I} \left|u_n(x) - g(x)\right| = \left\{
\begin{array}{lcl}
	\lim\limits_{n \to \infty} \sup \left[ \frac{n}{2}(x-\xi+\frac{1}{2n})^2 \right] & \xi-1/2n < x \leq \xi \\
	\lim\limits_{n \to \infty} \sup \left[ \frac{n}{2}(x-\xi-\frac{1}{2n})^2 \right] & \xi < x < \xi+1/2n
\end{array}
\right.
$$
$\Longrightarrow$
$$
	\lim\limits_{n \to \infty} \sup_{x \in [0,1]}\left|u_n(x) - g(x;\xi)\right| =\lim\limits_{n \to \infty} \frac{1}{2n} = 0
$$
Both the maximum values of the RHS go to 0, so the convergence is uniform.

\section*{Exercise 1.2}
\subsection*{i)}
It is a distribution. \\
Linearity: $T(\lambda_1 \varphi_1 + \lambda_2 \varphi_2)=(\lambda_1 \varphi_1 + \lambda_2 \varphi_2)(-10)=\lambda_1 \varphi_1(-10) + \lambda_2 \varphi_2(-10)=\lambda_1 T\varphi_1 + \lambda_2 T\varphi_2 $ \\
Continuity: $\left|\varphi(-10)\right| \leq \sup\limits_{x \in R^n} \left|\varphi_m(x)\right| \to 0$, which implies T to be continuous
\subsection*{ii)}
It is not a distribution. \\
Because $(\lambda \varphi_1(0) + \mu \varphi_2(0))^2 \neq \lambda \varphi_1(0)^2 + \mu \varphi_2(0)^2$, which implies non-linearity.
\subsection*{iii)}
It is a distribution. \\
Linearity: $T(\lambda_1 \varphi_1 + \lambda_2 \varphi_2)=grad(\lambda_1 \varphi_1 + \lambda_2 \varphi_2)(0)=grad(\lambda_1 \varphi_1)(0) + grad(\lambda_2 \varphi_2)(0)=\lambda_1 T\varphi_1 + \lambda_2 T\varphi_2$ \\
Continuity: $\sup\limits_{x \in  \mathbb{R}^n} \left|D^\alpha \varphi_m (x)\right| \xrightarrow{m \to \infty}0 $, and first order derivative automatically satisfies
\subsection*{iv)}
It is a distribution. \\
Linearity: follows i) and the property of point-wise addition\\
Continuity: $\left|T\varphi_m\right| =\left|\sum_{n=0}^\infty \varphi_m(n)\right|=R\cdot max\{\left|\varphi_m(n)\right|\} \xrightarrow{m \to \infty}0$
\subsection*{v)}
It is a distribution.\\
Linearity: follows the linearity of point-wise addition and integral operator \\
Continuity: $\left|T\varphi_m\right| = \left|\int_{S^{n-1}} \varphi_m \right| \xrightarrow{m \to \infty}0 $
\subsection*{vi)}
a) \\
It is not a distribution.\\
Because integral of $1/x$ goes to infinity in the vicinity of zero.\\
\\
b) \\
It is a distribution. \\
Linearity: $T(\lambda_1 \varphi_1 + \lambda_2 \varphi_2)= \lambda_1 \int_{\mathbb{R}^n}\frac{1}{\sqrt{|x|}}\varphi_1(x)dx + \lambda_2 \int_{\mathbb{R}^n}\frac{1}{\sqrt{|x|}}\varphi_2(x)dx=\lambda_1 T\varphi_1 + \lambda_2 T\varphi_2$ \\
Continuity: $\left|T\varphi_m\right| \xrightarrow{m \to \infty} \int_{\mathbb{R}}\frac{1}{\sqrt{|x|}} \left|\varphi_m(x)\right|dx \to \infty$\\
\\
c) \\
It is not a distribution. \\
Because the non-continuity similar to (a).
\end{document}
